\documentclass{book}

\usepackage[utf8]{inputenc}
\usepackage{titlesec}
\usepackage{easylist}
\usepackage{hanging}
\usepackage{hyperref}
\usepackage[a4paper,top=2.0cm,bottom=2.0cm,left=2.0cm,right=3.0cm]{geometry}
\usepackage{blindtext}
\usepackage{tipa}
\usepackage{epigraph}
\usepackage{enumerate}
\usepackage{longtable}
\usepackage{setspace}
\usepackage{verbatim}
\usepackage[T1]{fontenc}
\usepackage{graphicx}
\usepackage[italian]{babel}
\usepackage{amsmath}
\usepackage{pbox}
\usepackage{fancyhdr}
\usepackage{cancel}
\usepackage{tabularx}
\usepackage{booktabs}
\usepackage{multirow}
\usepackage{longtable}
\usepackage{tikz}
\usepackage{qtree}
\usepackage{tikz-qtree}
\usepackage{subfig}
\usepackage{xcolor}
\usepackage{amssymb}
\usepackage{mathrsfs}
\usepackage{textcomp}
\usepackage{imakeidx}
\usepackage{listings}
\usepackage{amsthm}
\usepackage{xpatch}
\usepackage{circuitikz}
\usepackage{tasks}
\usepackage{multicol}

% strutture di utilizzo comune 
\newtheorem{notab}{Nota bene}
\newtheorem{defi}{definizione}
\newtheorem{teo}{teorema}
\newtheorem{eser}{Esercizio}

\linespread{1.5} % l'interlinea

\frenchspacing

\newcommand{\abs}[1]{\lvert#1\rvert}

\usepackage{floatflt,epsfig}

\usepackage{multicol}
\newcommand\yellowbigsqcup[1][\displaystyle]{%
  \fboxrule0pt
  \ifx#1\textstyle\fboxsep-0.6pt\else\fboxsep-1.25pt\fi
  \mathrel{\fcolorbox{white}{yellow}{$#1\bigsqcup$}}}

\definecolor{codegreen}{rgb}{0,0.6,0}
\definecolor{codegray}{rgb}{0.5,0.5,0.5}
\definecolor{codepurple}{rgb}{0.58,0,0.82}
\definecolor{backcolour}{rgb}{0.95,0.95,0.92}

\lstdefinestyle{mystyle}{
    backgroundcolor=\color{backcolour},   
    commentstyle=\color{codegreen},
    keywordstyle=\color{magenta},
    numberstyle=\tiny\color{codegray},
    stringstyle=\color{codepurple},
    basicstyle=\ttfamily\footnotesize,
    breakatwhitespace=false,         
    breaklines=true,                 
    captionpos=b,                    
    keepspaces=true,                 
    numbers=left,                    
    numbersep=5pt,                  
    showspaces=false,                
    showstringspaces=false,
    showtabs=false,                  
    tabsize=2
}
\NewDocumentCommand{\codeword}{v}{%
\texttt{\textcolor{blue}{#1}}%
}
\lstset{language=C,keywordstyle={\bfseries \color{blue}}}
\lstset{language=java,keywordstyle={\bfseries \color{red}}}
\lstset{style=mystyle}

\title{Unix}
\author{Brian W. Kernighan \and Rob Pike}

\begin{document}
\maketitle
\tableofcontents
\lstlistoflistings
\chapter{L'UNIX per i principianti}
Cos'è l'UNIX? Letteralmente, si tratta del nucleo centrale di un sistema operativo
che utilizza la tecnica del time-sharing per gestire le risorse di un calcolatore e distri-
buirle tra i vari utenti. Il sistema consente ai singoli utenti di far girare i propri pro.
grammi, controlla le unità periferiche (dischi, terminali, stampanti e simili) collegate
alla macchina e mette a disposizione un sistema di archiviazione che dà la possibilità
di tenere a lungo in memoria informazioni quali programmi, dati e documenti.\\
In senso più ampio, spesso si intende per UNIX non solo il nucleo centrale, ma an
che i programmi essenziali quali i compilatori, i programmi per le attività di editing\footnote{Per "editing" si 
intende tutta quella serie di attività legate alla messa a punto redazionale di un testo prima della sua 
composizione definitiva per la stampa, i comandi di sistema, i programmi di utilità (per la co.
piatura e la stampa dei file) e così via. [N.d.T.]}
In senso ancora più ampio, l'UNIX può comprendere anche i programmi sviluppa-
ti dagli utenti, come ad esempio quelli per la preparazione dei documenti, routine per
l'analisi statistica e i pacchetti grafici.\\
L'accezione in cui il termine UNIX sarà usato dipenderà da quale livello del sistema sarà preso in considerazione. In questo libro, il contesto chiarirà il significato di
volta in volta.\\
Il sistema UNIX sembra a volte più difficile di quanto non lo sia in realtà, perché
in effetti, per i neofiti, è difficile saper usare in modo ottimale tutte le possibilità che
offre. Fortunatamente, però, non è difficile muovere i primi passi e basta conoscere
alcuni programmi per cominciare a lavorare. Con questo primo capitolo si vuole, nel
più breve tempo possibile, mettervi in grado di usare il sistema. Esso dà una visione
panoramica e non è un manuale; la materia verrà poi in gran parte trattata con maggior
dettaglio nei capitoli successivi.
Gli argomenti principali di questo capitolo sono i seguenti:
\begin{itemize}
\item attività di base: registrazioni in entrata e in uscita, comandi semplici, correzione
degli errori di battitura, spedizione della corrispondenza, comunicazioni tra ter-
minali.
\item operazioni quotidiane: i file\footnote{Un file è letteralmente un archivio di informazioni/dati.
 In questo testo useremo il termine inglese, comunemente impiegato in informatica). [N.d.T.]} e
  il sistema di gestione relativo, stampa dei file e aree di lavoro, comandi più comunemente usati.
\end{itemize}
Anche se decidete di leggere questo capitolo, vi servirà comunque avere a disposizione
 una copia del Manuale di programmazione UNIX; questo perché spesso è più facile
  per gli autori rimandarvi al manuale piuttosto che ripeterne i contenuti qui. Questo
libro, infatti, non vuole sostituirsi al manuale, ma intende spiegare come usare al meglio
 i comandi in esso descritti. Inoltre, potrebbero esservi delle differenze tra quanto
viene detto in questo testo e la realtà del sistema da voi usato. Il manuale riporta nelle
prime pagine un indice analitico, indispensabile per poter ritrovare i programmi più
adatti per la soluzione dei vari problemi; vi consigliamo di imparare ad utilizzarlo.\\
Infine, un piccolo suggerimento: non abbiate paura degli esperimenti. Sappiate
che, anche se siete dei principianti, sono pochi i possibili casi di errori che potrebbero
incidentalmente nuocere a voi o ad altri utenti. Perciò, cercate di imparare mettendo
in pratica da subito quello che leggete.\\
Questo è un capitolo lungo. Il miglior modo per affrontarlo è di leggerne solo qual
che pagina alla volta esperimentando subito quello che via via imparate.
\section{Come cominciare}
\subsection{ Alcuni prerequisiti a proposito di terminali e di dattilo scrittura}
Onde evitare di spiegare proprio tutto sull'uso dei calcolatori, diamo per scontato che
il lettore abbia famigliarità con i terminali e sappia come usarli. Qualora qualcosa di
quanto viene più avanti esposto vi risulti incomprensibile in base alla vostra esperienza,
 vi consigliamo di consultare un esperto del sistema da voi usato.\\
Il sistema UNIX è del tipo definito full duplex, vale a dire che i caratteri battuti sulla
 tastiera vengono inviati al sistema che a sua volta li rimanda al terminale perché
vengano scritti a video. Normalmente, questo processo, che ha le caratteristiche
dell'eco, fa sì che i caratteri siano copiati direttamente sul video, rendendo cosi possibile
 visualizzare immediatamente ciò che si batte sulla tastiera.\\
A volte, però, come nel caso in cui si debba battere una password (codice di accesso
 al sistema) segreta, il processo eco viene temporaneamente sospeso e i caratteri
battuti non appaiono a video. La maggior parte dei tasti riporta normali caratteri alfanumerici;
 però, ve ne sono alcuni che servono per dire al calcolatore come interpretare
  quanto viene battuto. Di gran lunga il più importante fra questi tasti è quello di
RETURN. Esso è praticamente un codice che segnala la fine di una riga di immissione;
 quando incontra questo codice, il sistema risponde posizionando immediatamente
 il cursore sul video alla riga successiva. È necessario premere il tasto RETURN perché
  il sistema interpreti i caratteri battuti.\\
Tale tasto è perciò un esempio di carattere di comando, vale a dire un carattere in-
visibile che serve a gestire alcuni aspetti di input o di output (immissione/emissione)
sul terminale. Su ogni terminale che si rispetti, per il RETURN esisterà un apposito
tasto; ma per la maggior parte degli altri caratteri di comando non sarà così. Infatti.
perché la macchina li senta, sarà necessario premere contemporaneamente il tasto
CONTROL (spesso abbreviato in CTL o CNTL o anche CTRL) e, a seconda dei casi
un altro tasto, generalmente una lettera dell'alfabeto. Ad esempio, il comando di r
torno si può dare premendo il tasto RETURN o, indifferentemente, premendo con-
temporaneamente CONTROL e la lettera "m''.Il comando RETURN può perciò
anche chiamarsi ``CONTROL - m'' (dal nome dei due tasti) e lo si può scrivere ctl-m
Citiamo qui altri caratteri di comando, per esempio "ctl-d", che serve per dire al pro.
gramma che l'input è terminato; ctrl-g, che fa suonare il campanello del terminale; ctl-h.
chiamato anche ritorno carattere, che può essere usato per correggere gli errori di
battitura; e infine ctrl-i, spesso chiamato tabulatore, per far avanzare il cursore direttamente
 alla tabulazione successiva, proprio come sulle normali macchine per scrivere.
  Le tabulazioni, sui sistemi UNIX, sono distanziate di 8 caratteri. Sia per il ritorno carattere sia per la tabulazione, esiste un tasto apposito sulla maggior parte dei termi
nali.\\
Vi sono poi altri due tasti che hanno uno speciale significato: il tasto DELETE
(cancella), talvolta chiamato anche RUBOUT o variamente abbreviato, e BREAK
(interruzione) chiamato a volte INTERRUPT. Nella maggior parte dei sistemi
UNIX, premendo il tasto DELETE si provoca l'immediato arresto di un programma
prima che sia finito. In alcuni altri sistemi, tale arresto si ottiene premendo ct/-c. E in
alcuni altri ancora, a seconda di come sono collegati i terminali, BREAK è sinonimo
di DELETE o ct/-c.
\subsection{Un esempio di collegamento in UNIX}
Vediamo di fare un esempio di dialogo tra voi e il vostro sistema UNIX.
In questo testo, ogniqualvolta faremo degli esempi pratici, useremo tre caratteri diversi
 che esemplifichiamo: messaggi di sistema, comandi e i testi dell'utente, commenti esplicativi\\
Effettuate il collegamento, sia componendo un numero telefonico oppure girando
un interruttore, a seconda dei casi. Dopo di che sul video dovrebbe apparire quanto
segue:
\begin{description}
\item[login: nome in codice] \footnote{In questo testo, per convenzione, il vostro nome in codice sarà "alba" e d'ora in poi lo useremo ogniqualvolta necessario. [N.d.T.].} Battete il vostro nome e quindi premete il tasto RETURN
\item[Password: ] La vostra password (codice di accesso segreto al sistema) non apparirà a video
mentre la battete
\item[You have mail: ] Avete della corrispondenza in arrivo. Potrete leggerla dopo aver effettuato le operazioni
di login
\item[\$ ] Quando sullo schermo appare il segno del dollaro significa che il sistema è pronto a
ricevere i vostri comandi
\item[\$ ] Premete due volte il tasto RETURN
\item[\$ date] Che data e che ore sono?\\
{\tt Sunday Sept 25, 23; 02: 57 EDT 1983}
\item[\$ vho] Chi sta usando la macchina?\\
\begin{tabular}{llr}
	jib & tty0 & Sep 25 13:59 \\
	alba &tty2 & Sep 25 23:01 \\
	mary & tty4 & Sep 25 19:03 \\
	doug & tty5 & Sep 25 19:22 \\
	egb & tty7 & Sep 25 17:17 \\
	bob & tty8 & Sep 25 20:48
\end{tabular}
\item[\$ mail ] Leggete i messaggi\\
{\tt Da doug Dom 25 Sett 20:53 EDT 1983}\\
{\tt chiamami quando vuoi lunedì}
\item[? ] Il comando RETURN fa apparire il
messaggio successivo
\item[Da mary - Dom. 25 Sett 19:07 EDT 1983 Si pranza a mezzogiorno domani?] Messaggio successivo
\item[? d ]  Cancellare questo messaggio
\item[\$ mail mary ]  Invia messaggio a mary\\
va bene per il pranzo alle 12
\item[ctl-d ] Fine messaggi
\item [\$ ] Chiudete il telefono o spegnete il terminale
e l'operazione è finita.
\end{description}
A volte ci si può collegare soltanto per leggere dei messaggi, come nel caso sopra indicato, altre volte ci si 
può collegare anche per lavorare. Nella parte rimanente di questo paragrafo prenderemo in esame il 
collegamento sopra descritto, oltre ad alcuni altri programmi che consentono di svolgere attività utili.
\subsection{Operazioni di collegamento}
Dovete in primo luogo farvi assegnare un codice di accesso e una password (codice
accesso segreto: d'ora in poi useremo il termine inglese, comune in ambiente informatico)
 dal coordinatore del sistema. L'UNIX è in grado di gestire molti tipi di terminali,
  ma è fortemente orientato verso macchine con caratteri minuscoli. La differenza
 tra le lettere maiuscole e minuscole è quindi importante con l'UNIX, perciò se il
vostro terminale scrive solo in maiuscolo potreste trovarvi in serie difficoltà e sarebbe
opportuno che vi procuraste un altro tipo di terminale.\\
Assicuratevi che gli interruttori siano nella corretta posizione di lavoro: maiuscolo
o minuscolo, full duplex ed eventuali altre impostazioni iniziali consigliate dal coordinatore:
 ad esempio la velocità di linea (cioè il numero di baud). Attivate il collegamento
  compiendo il gesto fatato previsto dal vostro terminale: sia esso la composizione
 di un numero telefonico o la semplice accensione di un interruttore. Al compi-
mento di tale gesto, a video dovrebbe apparire la scritta login:\\
Se il sistema dovesse inviare altri messaggi sconnessi, potrebbe significare che avete
impostato una velocità sbagliata; controllate quindi la velocità e anche gli altri inter-
ruttori. Se dal controllo risulta che tutto era stato eseguito correttamente, premete al-
cune volte, lentamente, il tasto BREAK oppure INTERRUPT.\\
Se, nonostante tutti questi accorgimenti, non riuscite ad ottenere la scritta login a
video, allora avete bisogno dell'aiuto di un tecnico. Una volta ottenuto login sullo
schermo, battete il vostro codice di accesso in lettere minuscole, e immediatamente
dopo premete RETURN. Se è necessaria la password, il sistema ve la chiederà, però,
come abbiamo già detto, non la farà apparire a video mentre la battete.\\
Dopo che avrete correttamente effettuato le operazioni di accesso, i vostri sforzi
saranno premiati con un messaggio a video, generalmente costituito da un unico carattere,
 indicante che il sistema è pronto ad accettare i vostri comandi. Tale messaggio 
 può essere costituito dal segno del dollaro (\$) oppure da quello di percentuale (\%)
ma, se volete cambiarlo come meglio preferite vi diremo come fare più avanti. Tale
lettera è in effetti stampata grazie al già menzionato programma interprete, che è
l'interfaccia principale col sistema.\\
Prima del messaggio suddetto, il sistema potrebbe automaticamente stampare la
data, oppure darvi una segnalazione di posta in arrivo. Il sistema vi può anche chiedere
quale tipo di terminale state usando, perché, sapendolo, sarà in grado di utilizzare
al meglio le caratteristiche tecniche del vostro terminale.
\subsection{Comandi di battitura}
Quando il sistema vi segnala di essere pronto (supponiamo con il segno del dollaro (\$)
come sopra citato) potete battere dei comandi, che non sono altro che degli ordini per
il sistema stesso. (Vi segnaliamo che in questo testo useremo il termine programma
come sinonimo di comando). Perciò, quando sullo schermo compare il segno del dollaro
dovete battere "date" (data) e premere il tasto RETURN. Il sistema dovrebbe rispondere
scrivendo la data e l'ora, e dandovi subito dopo un altro segno di dollaro;
per chiarire, tutto quanto sopra descritto sarà visualizzato cosi sul vostro terminale:
\begin{description}
\item[\$ ] date \\
{\tt Mon Sep 26 12:20:57 EDT 1983}
\item[\$ ]
\end{description}
Non dimenticate di premere RETURN e non battete voi il segno del dollaro. Se vede.
te che il sistema è lento a rispondere, premete il tasto RETURN; vedrete che qualcosa
succederà. Attenzione che d'ora in avanti non ripeteremo più di battere il tasto RETURN.
Ricordate però che bisogna premerlo ad ogni fine riga. 
il comando successivo è who (chi) per chiedere chi sono le persone collegate in quel
momento:
\begin{description}
\item[\$ ] who\\ 
\begin{tabular}{llr}
	rim & tty0 & Sep 26 11:17 \\
	pjw &tty4 & Sep 26 11:30 \\
	gerard & tty7 & Sep 26 10:27 \\
	mark & tty9 & Sep 26 07:59 \\
	alba & ttya & Sep 25 12:20 \\
\end{tabular}
\item[\$ ]
\end{description}
Nella prima colonna è indicato il nome dell'utente. Nella seconda. il nome che il sistema dà al tipo di collegamento usato ("tty" sta per "teletype", un sinonimo arcaico della parola "terminale"). Le restanti colonne danno la data e l'ora del collegamento. Potreste anche provare il seguente comando:
\begin{description}
\item[\$ who am I ]  (chi sono io?)\\
{\tt alba ttya Sep 26 12:20}
\item[\$]
\end{description}
Se, per un errore di battitura, scrivete un comando inesistente, il sistema vi informerà
di non averlo trovato, cosi:
\begin{description}
\item[\$ whom ]  Testo del comando scorretto (avete scritto
"whom" al posto di "who")
\item[whom: not fount] \dots e il sistema ha sapere come gestirlo 
\end{description}
Ovviamente, se per un errore di battitura scrivete il nome di un comando esistente,
che non è quello voluto in quel momento, esso verrà eseguito ugualmente, ma il risultato
non sarà quello desiderato.
\subsection{Comportamento strano del terminale}
Potrà capitare, a volte, che il vostro terminale si comporti in modo alquanto strano:
per esempio, potrebbe stampare ogni lettera due volte oppure, anche premendo il tasto
RETURN, il cursore non si posizionerà a margine della riga successiva. Per ovviare
a questo malfunzionamento è sufficiente spegnere e riaccendere il terminale,
oppure scollegarsi e quindi ricollegarsi. In alternativa, potete anche leggere la descrizione
 del comando stty ("set terminal options" : impostare le opzioni del terminale)
al paragrafo 1 del manuale. Se il vostro terminale non ha il tasto per le tabulazioni
potete ottenerle lo stesso battendo il comando:
\begin{description}
\item[\$ stty -tabs ] 
\end{description}
e il sistema convertirà i caratteri che gestiscono le tabulazioni nell'esatto numero di
spazi da voi voluti. Invece se con il terminale è possibile far impostare le tabulazioni
dal calcolatore, il comando tabs vi consentirà di ottenere i risultati voluti. Perché fun
zioni dovrete magari scrivere:
\begin{description}
\item[\$ tabs ] tipo di terminale usato 
\end{description}
(Vedere a questo proposito la descrizione del comando tabs nel manuale di programmazione).
\subsection{Errori di battitura}
Se fate un errore di battitura e ve ne accorgete prima di aver premuto il tasto RETURN,
 avete due modi per rimediare: potete cancellare i caratteri sbagliati uno alla
volta oppure annullare l'intera riga e ribatterla.
Se battete il carattere per l'annullamento della riga, per default\footnote{Default è un termine comunemente
usato in ambiente informatico per indicare un comportamento standard 
del sistema in assenza di istruzioni diverse. [N.d.T.].} il segno @ a esso
farà si che l'intera riga venga eliminata proprio come se non fosse mai stata battuta, e
vi riposizionerà su una nuova riga:
\begin{description}
\item[\$ ddtae@ ] Riga completamente sbagliata; ricominciare di nuovo
\item[ date ]  su un'altra riga\\
{\tt Mon Sep 26 12:23:39 EDT 1983}
\item[\$ ]
\end{description}
Il segno \# cancella l'ultimo carattere battuto; ogni \# cancella perciò un carattere alla
volta fino ad arrivare all'inizio della riga (ma non va oltre). Così, se fate piccoli errori
di battitura, potete correggerli via via:
\begin{description}
\item[\$ dd \#atte \#\#e] Correggete via via 
{\tt Mon Sep 26 12:24:02 EDT 1983}
\end{description}
I segni che provocano la cancellazione dei singoli caratteri o l'annullamento dell'inte.
ra riga sono estremamente dipendenti dal sistema. In molti sistemi (incluso quello che
usiamo noi) quello per la cancellazione dei singoli caratteri è stato sostituito dal tasto
di ritorno carattere e funziona piuttosto bene con i terminali video. Potete veloce.
mente controllare come stanno le cose con il sistema da voi usato nel modo seguente:
\begin{description}
\item[ datee \textleftarrow ]  su un'altra riga\textleftarrow\\
{\tt datee \textleftarrow: not found} Il tasto di ritorno non funziona\\ 
\item[\$ detee \# ] Provate il segno \#\\
{\tt Mon Sep 26 12:26:08 EDT 1983} È il segno \# quello che va bene
\item[\$ ]
\end{description}
(Come potete vedere, abbiamo usato il segno - per simboleggiare il ritorno carattere
perché poteste vederlo). Un altro modo molto comune per ordinare l'annullamento
di una riga è quello di battere ct/-u.\\
Nella rimanente parte di questo paragrafo useremo il simbolo \# come carattere di
cancellazione perché è chiaramente visibile. Cercate però di ricordare sempre qual è il
simbolo previsto dal vostro sistema, nel caso fosse diverso. Più avanti, nella parte dedicata 
alla personalizzazione del sistema, vi diremo come fare per usare come simbolo 
di cancellazione carattere o riga il segno che volete una volta per tutte.\\
Cosa fare nel caso in cui il simbolo che serve per il comando di cancellazione carattere,
o annullamento riga, dovesse essere scritto come parte integrante di un testo anziché 
come carattere di comando? Si dovrà semplicemente far precedere i simboli \# e
@ da una barra rovesciata, così:\\
Per inserire quindi in un testo i simboli \# oppure @ con il loro significato originario,
 bisognerà battere \textbackslash{} \# o @. A volte capita che il sistema, dopo che avete
battuto il carattere @ preceduto dalla barra rovesciata, faccia comunque avanzare il
cursore del terminale alla riga successiva. Questo non è grave, perché il sistema ha
comunque registrato il segno @ come da voi desiderato.\\
La barra rovesciata, talvolta chiamata "escape character" (carattere di cambio codice),
nella maggior parte dei casi viene usata per indicare che il carattere che la segue
è in qualche modo speciale. Per cancellare tale barra bisogna battere due volte il carattere
di cancellazione, così: \#\#. Vi rendete conto del perché?\\
Dovete sapere che i caratteri battuti, prima di arrivare a destinazione, sono esaminati
 ed interpretati da una sequenza di programmi e la loro esatta interpretazione dipende
  non solo dal punto d'arrivo, ma anche da come sono giunti a tale meta.\\
Ogni carattere battuto viene immediatamente rimandato, con effetto eco, al terminale,
 ad eccezione dei casi in cui l'effetto eco sia stato sospeso, il che è raro. Finché
non viene premuto il tasto RETURN, i caratteri sono temporaneamente memorizzati
dal nucleo centrale del sistema e perciò gli errori di battitura possono essere corretti
con i comandi di cancellazione carattere o annullamento riga. Quando uno dei due
caratteri suddetti è preceduto dalla barra rovesciata, il sistema non prende in considerazione 
la barra e memorizza invece i due simboli senza interpretarli come caratteri di
comando.\\
Quando premete il tasto RETURN, i caratteri memorizzati dal nucleo centrale del
sistema vengono inviati al programma che sta effettuando la lettura dal terminale
Tale programma può, a sua volta, interpretare in modo particolare i caratteri che riceve;
 per esempio, l'interprete non assegnerà nessun significato speciale a un carattere
 che sia preceduto da una barra rovesciata. Ma riprenderemo questo discorso al capitolo 3.
  Per il momento dovete solo ricordare che il nucleo centrale elabora i simboli di cancellazione carattere o annullamento riga, nonché la barra rovesciata, soltanto
se quest'ultima li precede; eventuali caratteri seguenti possono essere interpretati an
che da altri programmi.
\begin{eser}
	Spiegate cosa succede quando a video compare quanto segue:\\
	{\tt \$ date @}
\end{eser}
\begin{eser}
	La maggior parte dei programmi interprete (ma non la settima edizione) considera il simbolo \#
	come introduttivo di un commento, e ignora quindi tutta la parte di testo che va dal simbolo \# stesso
	alla fine della riga. Alla luce di quanto detto, spiegate quanto segue, assumendo che anche il 
	simbolo usato come comando di cancellazione sia \#:
	\begin{description}
		\item[\$ date ]
			{\tt Mon Sep 26 12:39:56 EDT 1983}
		\item[\$  \# date ] 
			{\tt Mon Sep 26 12:40:21 EDT 1983}
		\item[\$ \textbackslash{} \# date]
		\item[\$ \textbackslash{} \textbackslash{} \# date] \#date: not found
		\item[\$ ]		
	\end{description}
\end{eser}
\subsection{Immissione/emissione contemporanea di testi}
Il programma base, o nucleo centrale, legge quello che battete nel momento stesso in
cui lo fate, anche se sta effettuando altre operazioni; cosi potete battere alla velocità
che volete, ogniqualvolta lo volete, anche quando il sistema sta stampando a video
qualcosa per voi. Se durante un lavoro di battitura il sistema stesse svolgendo un lavoro
 di emissione impegnando il video, i vostri caratteri appariranno sullo schermo
intercalati con quelli inviati dal sistema, ma saranno comunque registrati separatamente
 e interpretati in modo corretto. Potete inoltre battere dei comandi uno dopo
l'altro, senza aspettare, prima di battere il comando successivo, che il primo abbia
eseguito il suo compito, o anche prima ancora che lo incominci.
\subsection{Come arrestare un programma}
La maggior parte dei comandi può essere interrotta battendo il comando DELETE
(distruggi). Lo si può fare anche premendo il tasto BREAK che si trova sulla maggior
parte delle tastiere, benché ciò sia dipendente dal sistema usato. In alcuni programmi,
 come quelli per l'editing dei testi, l'immissione di DELETE provocherà l'interruzione
 di qualunque operazione in corso, ma senza farvi uscire dal programma.
La maggior parte dei programmi si interrompe quando si spegne il terminale o si
appende il microtelefono. Se volete arrestare temporaneamente un lavoro di emissione
 in modo da trattenere sullo schermo una parte importante del testo prima che 
scompaia dalla vista, dovete battere ctl-s. L'output si interromperà quasi immediata-
mente; il programma resterà quindi sospeso finché non lo riprenderete. Per riprenderlo
 dovete battere ctl-q.
\subsection{Come scollegarsi}
Il modo migliore per uscire dal sistema è di battere ct/-d invece di dare un comando; in
tal modo si segnala all'interprete che l'input è finito. (Come quanto sopra effettivamente
 avviene, vi verrà spiegato nel capitolo successivo). Generalmente, potete semplicemente
 spegnere il terminale o appendere il microtelefono, ma se con queste operazioni
  vi scollegate veramente dipenderà dal sistema che state usando.
\subsection{Corrispondenza}
Il sistema consente di corrispondere con altri utenti, cosi a volte, quando vi collegate,
potrà capitarvi di vedere il seguente messaggio: you have mail (c'è della corrispondenza)
prima che sullo schermo compaia il simbolo indicante che il sistema è pronto a ricevere
i vostri comandi. Per leggere i messaggi dovete battere:
\begin{description}
\item[\$ mail] 
\end{description}
i messaggi verranno stampati uno alla volta, e il primo di essi sarà quello pervenuto
per ultimo. Dopo ogni messaggio, il sistema aspetterà che gli diciate cosa fare. Le due
risposte fondamentali sono d, che cancella il messaggio e RETURN, che non lo can
cella (cosicché lo ritroverete la prossima volta che leggerete i messaggi in arrivo). Si
possono dare altre risposte, come p, per ristampare un messaggio, s nome del file per tenerlo
 nel file che avete indicato, e q per uscire dal programma mail (corrispondenza).
Se non avete chiaro il concetto di file, pensate a un luogo dove potete conservare delle
informazioni, alle quali avete assegnato un titolo a vostra scelta, per poi ritrovarle
quando vi servono. I file sono l'argomento del paragrafo 1.2 e, in effetti, di gran parte 
di questo libro.\\
Il programma mail è tra quelli che probabilmente troverete diversi nel vostro sistema
 da come lo descriviamo qui, perché ne esistono molte varianti. Consultate il vostro
 manuale per i dettagli.\\
Spedire la corrispondenza a qualcuno è molto semplice. Supponete di dover mandare
un messaggio a una persona che abbia come nome di utente nico. Il modo più facile
di procedere è il seguente.
\begin{description}
\item[\$ mail nico] 
\end{description}
{\it Ed ora battete il testo della lettera usando quante righe volete. Dopo l'ultima riga dovete battere una "d"
di controllo. }
\begin{description}
\item[\$ ctl-d]
\item[\$ ] 
\end{description}
Il comando ct/-d indica la fine del messaggio e segnala alla funzione mail che l'input è
finito. Se, mentre state scrivendo una lettera, cambiate idea e decidete di non spedir
la, premete il tasto DELETE invece di ct/-d. La lettera lasciata a metà verrà memorizzata
 in un file chiamato dead. letter (lettere) invece di venire spedita.
Per fare esercizio, potete spedire dei messaggi a voi stessi, e poi battere mail per leggerli.
 (Questo tipo di operazione non è così strano come potrebbe sembrare; è invece
un facile mezzo per ricordare le operazioni da svolgere).\\
Vi sono anche altri modi per inviare dei messaggi: per esempio si può spedire una
lettera preparata in precedenza, oppure uno stesso testo a più destinatari, e si possono 
anche mandare messaggi a utenti di altre macchine. Per maggiori dettagli, si rinvia
alla descrizione del comando mail al paragrafo 1 del Manuale di programmazione
UNIX. D'ora in poi useremo la notazione mail (1) per indicare la pagina che descrive 
la funzione mail al paragrafo 1 del manuale. Tutti i comandi di cui abbiamo parlato in
questo capitolo si trovano al paragrafo 1 del manuale.\\
Ci potrebbe essere anche la possibilità di fare l'agenda (vedere calendar (1)); vi spiegheremo
al capitolo 4 come fare, se non lo sapete già.
\subsection{Dialoghi tra gli utenti}
Se il sistema UNIX da voi usato è collegato con più utenti, vi potrà capitare di vedere
un giorno sul vostro terminale un messaggio di questo tipo:
\begin{multicols}{2}
	Message from mary tty7\dots\\
	(Messaggio da mary tty7\dots)
\end{multicols}
accompagnato da un segnale acustico di avvertimento. Questo significa che l'utente
Mary vuole dialogare con voi, ma a meno che voi non eseguiate delle esplicite operazioni,
 non vi sarà possibile mettervi in contatto. Per attivare la comunicazione dovete
battere:
\begin{multicols}{2}
	\$ write mary\\
\end{multicols}
In questo modo si attiva un canale di comunicazione a due vie. E così, quanto Mary
scrive dal suo terminale apparirà sul vostro schermo e viceversa, anche se il canale sarà
 lento, un po' come parlare alla Luna.\\
Quando operate nell'ambito di un programma, dovete mettervi nella condizione di
poter dare dei comandi. Generalmente, qualunque tipo di programma si fermerà o
verrà fatto fermare, ma ve ne sono alcuni, come quello per le operazioni di editing e
lo stesso write, che hanno un comando "!" che consente di ritornare temporaneamente
 nell'interprete (vedere la tabella 2 dell'appendice 1).\\
Il comando write non prevede regole speciali, cosicché dovrete organizzare voi un
protocollo al fine di tenere separati i messaggi in partenza da quelli in arrivo da Mary.
 Un modo per organizzarsi è di prevedere dei turni, chiudendo ogni turno con il
simbolo o, che significa "over"' (passo), e di segnalare l'intenzione di uscire dalla
funzione con la scritta (o0) che significa "'over and out' (passo e chiudo).

\end{document}